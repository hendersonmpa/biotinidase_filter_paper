% Created 2021-12-09 Thu 10:39
% Intended LaTeX compiler: pdflatex
\documentclass[review]{elsarticle}
\usepackage[utf8]{inputenc}
\usepackage[T1]{fontenc}
\usepackage{graphicx}
\usepackage{longtable}
\usepackage{wrapfig}
\usepackage{rotating}
\usepackage[normalem]{ulem}
\usepackage{amsmath}
\usepackage{amssymb}
\usepackage{capt-of}
\usepackage{hyperref}
\usepackage{lineno}
\linenumbers
\usepackage{setspace}
\onehalfspacing
\authblk
\usepackage{pdfpages}
\usepackage{textpos}
\usepackage[final]{draftwatermark}
\usepackage{amsmath}
\usepackage{chemfig}
\setchemfig{atom style={scale=0.45}}
\usepackage[]{mhchem}
\date{}
\title{}
\hypersetup{
 pdfauthor={Matthew Henderson},
 pdftitle={},
 pdfkeywords={},
 pdfsubject={},
 pdfcreator={Emacs 27.2 (Org mode 9.5)}, 
 pdflang={English}}
\usepackage[notquote]{hanging}
\begin{document}\makeatletter
\newcommand{\citeprocitem}[2]{\hyper@linkstart{cite}{citeproc_bib_item_#1}#2\hyper@linkend}
\makeatother



\begin{frontmatter}
\title{Collection of Dried Blood Spot Samples on Recently Printed Filter Paper Cards Inhibits Biotinidase Activity}
\author[NSO]{Matthew P.A. Henderson\corref{cor1}}
\ead{mhenders@cheo.on.ca}
\author[NSO]{Nathan McIntosh}
\author[NSO]{Amy Chambers}
\author[NSO]{Emily Desmoreaux}
\author[NSO]{Michael Kowalski}
\author[NSO]{Jennifer Milburn}
\author[NSO]{Pranesh Chakraborty}
\address[NSO]{Newborn Screening Ontario, Children's Hospital of Eastern Ontario, 415 Smyth Rd, Ottawa Ontario}
\cortext[cor1]{Corresponding author}
\end{frontmatter}

\section*{STARTED Abstract}
\label{sec:org73e056a}
\begin{description}
\item[{Background}] From 2016 to 2017 the annual screen positive rate for
biotinidase deficiency in Ontario more than doubled from
2.65x10\textsuperscript{-4} to 6.57x10\textsuperscript{-4}.

\item[{Methods}] A data mining approach was used to investigate the role
of temperature, filter paper lot, and print run on the screen
positive rate for biotinidase deficiency. The laboratory information
system was queried for all biotinidase activity results for the
period 2014-01-12 to 2018-09-27 (n = 678,354). During the study
period biotinidase activity was measured on the SpotCheck Pro
platform (Astoria Pacific, Oregon USA.) with results reported in
microplate response units (MRU).

\item[{Results}] Decomposition was used to separate seasonality from an
underlying trend in the 4 year time series of biotindase activity
results. This analysis revealed a marked seasonal effect (winter =
mean + <= 16 MRU, summer = mean - <= 19 MRU) and a profound
non-linear negative trend (max = 126.7 MRU, min = MRU). External
temperature was correlated with biotinidase results (Pearson's r =
0.79) but not with the observed negative trend (Pearson's r =
-0.06). Time series analysis of biotinidase results grouped by print
lot of filter paper revealed that recently printed filter paper
cards exhibit a time dependent inhibition of biotinidase activity
(observed in seven print lots over four years). This inhibition
resolved in approximately 3 months (Lot A: mean result 0-3.5 months
= 66.0 MRU, mean result 3-6 months = 90.3 MRU).

\item[{Conclusions}] In Ontario, lots of filter paper collection cards are
generally issued once a year. Due to operational changes in 2017
four filter paper collection cards lots were issued. These new
filter paper lots caused a pronounced negative trend in biotinidase
activity leading to an increased biotinidase deficiency screen
positive rate in 2017.
\end{description}

\section*{Keywords}
\label{sec:orgc0d6044}
Newborn Screening, Biotinidase Deficiency, Dried Blood Spots, Filter
Paper, Seasonal Variation, Enzyme Activity
\section*{{\bfseries\sffamily TODO} Introduction}
\label{sec:org2cc3f38}
\begin{itemize}
\item Biotinidase deficiency is an autosomal recessive disorder of biotin
recycling associated with secondary alterations in amino acid,
carbohydrate, and fatty acid metabolism (Figure \ref{fig:org969d190}).

\item Biotinidase and newborn screening
\item Free biotin, whose primary source is the diet, is
actively transported across the intestinal membrane and blood–brain
barrier. This free biotin pool provides biotin to the various
apocarboxylases (pyruvate carboxylase, acetyl-CoA carboxylase,
propionyl-CoA carboxylase, and beta-methylcrotonyl-CoA
carboxylase). Holocarboxylase synthetase assembles active
holocarboxylases by covalently binding biotin to their active
sites. The carboxylases are important for gluconeogenesis, fatty
acid synthesis, and the catabolism of several branch-chain amino
acids. Eventually, the holocarboxylases are degraded proteolytically
to biocytin or small biotinyl-peptides. These compounds are further
cleaved by biotinidase to produce lysine and free biotin, which can
enter the free biotin pool, thereby recycling the vitamin. ACC,
acetyl-CoA carboxylase; AMP, adenosine monophosphate; ATP, adenosine
triphosphate; MCC, 3-methylcrotonyl carboxylase; PC, pyruvate
carboxylase; PCC, propionyl-CoA carboxylase; PPi, inorganic
pyrophosphate.
\item Biotinidase deficiency is caused by absent or markedly deficient
activity of biotinidase.

\item Biotinidase is a cytosolic enzyme that liberates biotin from
biocytin during the normal proteolytic turnover of holocarboxylases
and other biotiny-lated proteins.

\item Based on newborn screening outcome data from 2006, the incidence of
profound biotinidase deficiency in the United States is estimated at
1/80,000 births, and partial biotinidase deficiency between 1/31,000
and 1/40,000 (\hyperlink{citeproc_bib_item_4}{Strovel et al. 2017}).

\item Diagnosis of biotinidase deficiency is based on demonstrating
deficient enzyme activity in serum or plasma

\item Patients with profound biotinidase deficiency have less than 10\% of
mean normal serum activity (\hyperlink{citeproc_bib_item_4}{Strovel et al. 2017}).

\item Patients with partial biotinidase deficiency have 10-30\% of mean
normal serum activity (\hyperlink{citeproc_bib_item_4}{Strovel et al. 2017}).
\end{itemize}


\begin{figure}[htbp]
\centering
\includegraphics[width=.9\textwidth]{./figures/biot_cycle.png}
\caption{\label{fig:org969d190}The Biotin Cycle (\hyperlink{citeproc_bib_item_4}{Strovel et al. 2017})}
\end{figure}

\section*{Material and Methods}
\label{sec:org9eeeb96}
A data mining approach was used to investigate the affect of
temperature and filter paper lots on the screen positive rate for
biotinidase deficiency. The laboratory information system was queried
for all newborn screening biotinidase activity results from the first
phase of screening for the period 2014-01-12 to 2019-07-27 (n =
798,770). During the study period biotinidase activity was measured on
the SpotCheck Pro platform (Astoria Pacific, Oregon USA.). Biotinidase
activity was reported in microplate response units (MRU).

Daily mean temperature data for weather station 42183 in Barrie
Ontario was downloaded from Environment Canada. Time series
decomposition was used to separate seasonality from an underlying
trend in the 4 year time series of biotindase activity results.

Whole blood samples were heat treated to reduce biotinidase
activity. The heat treated whole blood was then mixed with untreated
whole blood in a serial dilution to create a set of samples with a
range of biotinidase activity. Biotinidase acitivty in these samples
was measured after spotting onto filter paper lots w152, w161, w162
and 171 and allowing to dry at room temperature overnight.

The R language for statistical computing was used for all data
analysis with extensive use of tidyverse packages for data
manipulation, mcr deming regression, xts and tsa for time series
analysis (\hyperlink{citeproc_bib_item_5}{Team 2020}; \hyperlink{citeproc_bib_item_6}{Wickham et al. 2019}; \hyperlink{citeproc_bib_item_2}{Fabian Model <fabian.model@roche.com> 2014}; \hyperlink{citeproc_bib_item_3}{Ryan and Ulrich 2020}; \hyperlink{citeproc_bib_item_1}{Chan and Ripley 2020}).

\section*{{\bfseries\sffamily TODO} Results}
\label{sec:org37d2752}
\subsection*{Time Series Analysis}
\label{sec:org25be887}
Time series analysis was performed to examine trends in biotinidase
and GALT activity measured for newborn screening over the four year
study period. As expected both analytes demonstrate seasonal variation
in measured activity (Figure \ref{fig:orge7a9de2} and \ref{fig:org571a718}). As a result
the screen positive rate for biotinidase deficiency and galactosemia
is increased in the summer months. However there is a period in early
2017 during which the screen positive rate for biotinidase deficiency
increases despite external temperatures below zero (Figure \ref{fig:orge7a9de2}
between blue lines). There is no corresponding change in GALT screen
positive rate during this time period (Figure \ref{fig:orge7a9de2}
between blue lines). 

% latex table generated in R 4.0.3 by xtable 1.8-4 package
% Thu Dec  9 10:39:35 2021
\begin{table}[ht]
\centering
\begin{tabular}{rrrrr}
  \hline
year & n & median & pos & rate \\ 
  \hline
2014 & 140620 & 119.87 &  78 & 0.00055 \\ 
  2015 & 140812 & 122.93 &  40 & 0.00028 \\ 
  2016 & 143361 & 120.25 &  38 & 0.00027 \\ 
  2017 & 144524 & 105.31 &  95 & 0.00066 \\ 
  2018 & 146365 & 111.90 &  88 & 0.0006 \\ 
  2019 & 83088 & 120.49 &  27 & 0.00032 \\ 
   \hline
\end{tabular}
\caption{Yearly Biotinidase Screen Positive Rate} 
\label{tab:biot_year}
\end{table}

% latex table generated in R 4.0.3 by xtable 1.8-4 package
% Thu Dec  9 10:39:35 2021
\begin{table}[ht]
\centering
\begin{tabular}{rrrrr}
  \hline
year & n & median & pos & rate \\ 
  \hline
2014 & 140678 & 8.37 &  20 & 0.00014 \\ 
  2015 & 140171 & 7.93 &  12 & 8.6e-05 \\ 
  2016 & 143352 & 8.13 &  21 & 0.00015 \\ 
  2017 & 143261 & 8.46 &  14 & 9.8e-05 \\ 
  2018 & 143592 & 8.22 &  13 & 9.1e-05 \\ 
  2019 & 82116 & 8.07 &   6 & 7.3e-05 \\ 
   \hline
\end{tabular}
\caption{Yearly Galactosemia Screen Positive Rate} 
\label{tab:galt_year}
\end{table}

\begin{figure}[htbp]
\centering
\includegraphics[width=\textwidth]{./figures/biotpts.pdf}
\caption{\label{fig:orge7a9de2}Time series of weekly screen positive biotinidase deficiency referrals, biotinidase activity and mean weekly temperature (\degree{}C)}
\end{figure}



\begin{figure}[htbp]
\centering
\includegraphics[width=\textwidth]{./figures/galtpts.pdf}
\caption{\label{fig:org571a718}Time series of weekly screen positive galactosemia referrals, GALT activity and mean weekly temperature (\degree{}C)}
\end{figure}

\clearpage

\subsection*{Seasonal Trend Decomposition}
\label{sec:orgd02bc2b}
Seasonal trend decomposition was used to identify trends in
biotinidase activity after seasonal adjustment (Figure
\ref{fig:org08ac55d}). Median weekly biotinidase activity revealed a marked
seasonal effect with high activity in the winter (median + \(\le\) 17 MRU)
and low activity in the summer (median - \(\le\) 20) in addtion to a
profound non-linear negative trend (Figure \ref{fig:org08ac55d}). External
temperature was correlated with biotinidase activity (Pearson's r =
0.81) but not with the observed negative trend (Pearson's r = 0.004).

\begin{figure}[htbp]
\centering
\includegraphics[width=\textwidth]{./figures/biotdecomp.pdf}
\caption{\label{fig:org08ac55d}Decomposition of the median weekly biotinidiase activity time series into seasonal, random and trend components.}
\end{figure}

\clearpage

\subsection*{Biotinidase active by Filter Paper Collection Card Lot}
\label{sec:org3f50ff7}
Time series analysis of median weekly biotinidase results grouped by
filter paper lot revealed that filter paper cards exhibit a time
dependent inhibition of biotinidase activity, observed in seven
filter paper lots over four years (Figure \ref{fig:org1b9e7e1}).

\begin{figure}[htbp]
\centering
\includegraphics[width=\textwidth]{./figures/biotform.pdf}
\caption{\label{fig:org1b9e7e1}Median weekly biotinidase activity by filter paper collection card lot.}
\end{figure}

\clearpage

\subsection*{Inhibition of Biotinidase Resolved After 2 Months}
\label{sec:org9f553f5}
Direct comparison of measured biotinidase activity for whole blood
samples spotted onto a paper lot (w171) recently received from the
printers with material in circulation shows a notable negative
bias (Figure \ref{fig:orgee61564}). Inhibition of biotindase activity resolved after two months
(Figures \ref{fig:org1b9e7e1} and \ref{fig:orgbc4c2bf}).

\begin{figure}[htbp]
\centering
\includegraphics[width=.9\linewidth]{./figures/spmat.pdf}
\caption{\label{fig:orgee61564}Comparison of biotinidase activity in samples collected simultaneously on filter paper lots w152, w161, w162, w171 at Time 0. linear regression- blue, line of identity- red}
\end{figure}

\begin{figure}[htbp]
\centering
\includegraphics[width=.9\linewidth]{./figures/demingw171.pdf}
\caption{\label{fig:org6510321}Biotinidase Activity on Filter Paper Lot W152 v W171 with Deming Regression at Time 0.}
\end{figure}
\clearpage

\begin{figure}[htbp]
\centering
\includegraphics[width=.9\linewidth]{./figures/spmat2.pdf}
\caption{\label{fig:orgbc4c2bf}Comparison of biotinidase activity in samples collected simultaneously on filter paper lots w152, w161, w162, w171 at Time 0 + 2 months. linear regression- blue, line of identity- red}
\end{figure}

\begin{figure}[htbp]
\centering
\includegraphics[width=.9\linewidth]{./figures/demingw171_2.pdf}
\caption{\label{fig:org5ce37a9}Biotinidase Filter Paper W152 v W117 Lot Comparison at Time 0 + 2 months.}
\end{figure}

\clearpage

\section*{{\bfseries\sffamily TODO} Conclusions}
\label{sec:org73d0702}

\begin{itemize}
\item In Ontario, new lots of filter paper collection cards are generally
issued once a year. Due to operational changes in 2017 three filter
paper collection cards lots were issued.
\item These new filter paper lots caused a pronounced negative trend in
biotinidase activity leading to an increased biotinidase deficiency
screen positive rate in 2017.
\end{itemize}

\section*{References}
\label{sec:org8b4f426}
\begin{hangparas}{1.5em}{1}
\hypertarget{citeproc_bib_item_1}{Chan, Kung-Sik, and Brian Ripley. 2020. \textit{TSA: Time Series Analysis}. \href{https://CRAN.R-project.org/package=TSA}{https://CRAN.R-project.org/package=TSA}.}

\hypertarget{citeproc_bib_item_2}{Fabian Model <fabian.model@roche.com>, Ekaterina Manuilova Andre Schuetzenmeister <andre.schuetzenmeister@roche.com>. 2014. \textit{Mcr: Method Comparison Regression}. \href{https://CRAN.R-project.org/package=mcr}{https://CRAN.R-project.org/package=mcr}.}

\hypertarget{citeproc_bib_item_3}{Ryan, Jeffrey A., and Joshua M. Ulrich. 2020. \textit{Xts: eXtensible Time Series}. \href{https://CRAN.R-project.org/package=xts}{https://CRAN.R-project.org/package=xts}.}

\hypertarget{citeproc_bib_item_4}{Strovel, Erin T., Tina M. Cowan, Anna I. Scott, and Barry Wolf. 2017. “Laboratory diagnosis of biotinidase deficiency, 2017 update: A technical standard and guideline of the American College of Medical Genetics and Genomics.” \textit{Genetics in Medicine} 19 (10). Nature Publishing Group. \href{https://doi.org/10.1038/gim.2017.84}{https://doi.org/10.1038/gim.2017.84}.}

\hypertarget{citeproc_bib_item_5}{Team, R Core. 2020. \textit{R: A Language and Environment for Statistical Computing}. Vienna, Austria: R Foundation for Statistical Computing. \href{https://www.R-project.org/}{https://www.R-project.org/}.}

\hypertarget{citeproc_bib_item_6}{Wickham, Hadley, Mara Averick, Jennifer Bryan, Winston Chang, Lucy D’Agostino McGowan, Romain François, Garrett Grolemund, et al. 2019. “Welcome to the tidyverse.” \textit{Journal of Open Source Software} 4 (43):1686. \href{https://doi.org/10.21105/joss.01686}{https://doi.org/10.21105/joss.01686}.}
\end{hangparas}
\end{document}