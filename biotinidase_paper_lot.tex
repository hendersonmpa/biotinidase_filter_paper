% Created 2021-12-14 Tue 20:31
% Intended LaTeX compiler: pdflatex
\documentclass[review]{elsarticle}
\usepackage[utf8]{inputenc}
\usepackage[T1]{fontenc}
\usepackage{graphicx}
\usepackage{longtable}
\usepackage{wrapfig}
\usepackage{rotating}
\usepackage[normalem]{ulem}
\usepackage{amsmath}
\usepackage{amssymb}
\usepackage{capt-of}
\usepackage{hyperref}
\usepackage{lineno}
\linenumbers
\usepackage{setspace}
\onehalfspacing
\authblk
\usepackage{pdfpages}
\usepackage{textpos}
\usepackage[final]{draftwatermark}
\usepackage{gensymb}
\usepackage{amsmath}
\usepackage{chemfig}
\setchemfig{atom style={scale=0.45}}
\usepackage[]{mhchem}
\date{}
\title{}
\hypersetup{
 pdfauthor={Matthew Henderson},
 pdftitle={},
 pdfkeywords={},
 pdfsubject={},
 pdfcreator={Emacs 27.2 (Org mode 9.5)}, 
 pdflang={English}}
\usepackage[notquote]{hanging}
\begin{document}\makeatletter
\newcommand{\citeprocitem}[2]{\hyper@linkstart{cite}{citeproc_bib_item_#1}#2\hyper@linkend}
\makeatother



\begin{frontmatter}
\title{Season and Filter Paper Collection Cards Affect Biotinidase Activity}
\author[NSO]{Matthew P.A. Henderson\corref{cor1}}
\ead{mhenderson@cheo.on.ca}
\author[NSO]{Nathan McIntosh}
\author[NSO]{Amy Chambers}
\author[NSO]{Emily Desormeaux}
\author[NSO]{Michael Kowalski}
\author[NSO]{Jennifer Milburn}
\author[NSO]{Pranesh Chakraborty}
\address[NSO]{Newborn Screening Ontario, Children's Hospital of Eastern Ontario, 415 Smyth Rd, Ottawa Ontario}
\cortext[cor1]{Corresponding author}
\end{frontmatter}

\section*{Abstract}
\label{sec:orgad2cbf1}
\begin{description}
\item[{Objective}] This study set out to examine preanlytical factors
affecting the frequency of positive results in newborn screening for
biotinidase deficiency. This is investigation was prompted by an
increase in annual screen positive rate for biotinidase deficiency
in Ontario from 2.65x10\textsuperscript{-4} in 2016 to 6.57x10\textsuperscript{-4} in 2017.

\item[{Design and Methods}] A data mining approach was used to investigate
the role of temperature, filter paper lot, and print run on the
screen positive rate for biotinidase deficiency. The laboratory
information system was queried for all biotinidase activity results
for the period 2014-01-12 to 2018-09-27 (n = 678,354). During the
study period biotinidase activity was measured on the SpotCheck Pro
platform (Astoria Pacific, Oregon USA.) with results reported in
microplate response units (MRU).

\item[{Results}] Decomposition was used to separate seasonality from an
underlying trend in the 4 year time series of biotindase activity
results. This analysis revealed a marked seasonal effect (winter =
mean + <= 16 MRU, summer = mean - <= 19 MRU) and a profound
non-linear negative trend (max = 126.7 MRU, min = MRU). External
temperature was correlated with biotinidase results (Pearson's r =
0.79) but not with the observed negative trend (Pearson's r =
-0.06). Time series analysis of biotinidase results grouped by print
lot of filter paper revealed that recently printed filter paper
cards exhibit a time dependent inhibition of biotinidase activity
(observed in seven print lots over four years). This inhibition
resolved in approximately 3 months (Lot A: mean result 0-3.5 months
= 66.0 MRU, mean result 3-6 months = 90.3 MRU).

\item[{Conclusions}] In Ontario, lots of filter paper collection cards are
generally issued once a year. Due to operational changes in 2017
four filter paper collection cards lots were issued. These new
filter paper lots caused a pronounced negative trend in biotinidase
activity leading to an increased biotinidase deficiency screen
positive rate in 2017.
\end{description}

\section*{Keywords}
\label{sec:orgf8d5148}
Newborn Screening, Biotinidase Deficiency, Dried Blood Spots, Filter
Paper, Seasonal Variation, Enzyme Activity
\section*{Introduction}
\label{sec:orgcce9725}
This study set out to examine pre-analytical factors affecting the
frequency of positive results in newborn screening for biotinidase
deficiency. This investigation was prompted by a marked increase in
annual screen positive rate for biotinidase deficiency in Ontario from
2.65x10\textsuperscript{-4} in 2016 to 6.57x10\textsuperscript{-4} in 2017.

Biotinidase deficiency is an autosomal recessive disorder caused by
absent or markedly deficient activity of biotinidase.  Biotinidase is
a cytosolic enzyme that liberates biotin from biocytin during the
normal proteolytic turnover of holocarboxylases and other biotinylated
proteins. The absence of appropriate biotin recycling is associated
with secondary alterations in amino acid, carbohydrate, and fatty acid
metabolism (Figure \ref{fig:org1b71e4a}).

Virginia was the first jurisdiction to perform universal newborn
screening for biotinidase deficiency, beginning in 1984 (\hyperlink{citeproc_bib_item_10}{Wolf 2015}). Biotinidase deficiency has been a target disorder
for newborn screening in Ontario since 2007 and was added to the
Recommended Universal Screening Panel in the United States in 2006 (\hyperlink{citeproc_bib_item_8}{Watson et al. 2006}).  Based on newborn screening outcome data from
2006, the incidence of profound biotinidase deficiency in the United
States is estimated at 1/80,000 births, and partial biotinidase
deficiency between 1/31,000 and 1/40,000 (\hyperlink{citeproc_bib_item_5}{Strovel et al. 2017}). Diagnosis of biotinidase deficiency is based on
demonstrating deficient enzyme activity in serum or plasma. Patients
with profound biotinidase deficiency have less than 10\% of mean normal
serum activity (\hyperlink{citeproc_bib_item_5}{Strovel et al. 2017}). Patients with partial biotinidase
deficiency have 10-30\% of mean normal serum activity (\hyperlink{citeproc_bib_item_5}{Strovel et al. 2017}).

Avoiding false negative newborn screening results is of paramount
importance in newborn screening for treatable disorders. There is
recognition that false positive newborn screening results are also an
adverse outcome of newborn screening and need to be reduced for the
benefit of families and the healthcare system (\hyperlink{citeproc_bib_item_3}{Kwon and Farrell 2000}). A
previous observational study examining of the affect of season on the
screen positive rate for biotinidase deficiency found no significant
seasonal effect, this may be due to differences in climate and
screening thresholds (\hyperlink{citeproc_bib_item_7}{Thibodeau et al. 1993}). Controlled experiments
demonstrate that exposure to high temperature and humidity decreases
biotinidase activity. Evidence presented here demonstrates that both
seasonal conditions and filter paper collection cards affect
biotinidase activity.


\begin{figure}[htbp]
\centering
\includegraphics[width=.8\textwidth]{./figures/cycle.pdf}
\caption{\label{fig:org1b71e4a}The Biotin Cycle adapted from (\hyperlink{citeproc_bib_item_5}{Strovel et al. 2017}). Grey ovals contain substrates and products. Labelled black points represent enzymes. Metabolic functions of holocarboxylases are listed in the octagon.}
\end{figure}

\section*{Material and Methods}
\label{sec:org0cfe420}
A data mining approach was used to investigate the affect of
temperature and filter paper lots on the screen positive rate for
biotinidase deficiency. The laboratory information system was queried
for all newborn screening biotinidase activity results from the first
phase of screening for the period 2014-01-12 to 2019-07-27 (n =
798,770). During the study period biotinidase activity was measured on
the SpotCheck Pro platform (Astoria Pacific, Oregon USA.). Biotinidase
activity was reported in microplate response units (MRU).

Daily mean temperature data for weather station 42183 in Barrie
Ontario was downloaded from Environment Canada. Time series
decomposition was used to separate seasonality from an underlying
trend in the 4 year time series of biotindase activity results.

Whole blood samples were heat treated to reduce biotinidase
activity. The heat treated whole blood was then mixed with untreated
whole blood in a serial dilution to create a set of samples with a
range of biotinidase activity. Biotinidase acitivty in these samples
was measured after spotting onto filter paper lots w152, w161, w162
and 171 and allowing to dry at room temperature overnight.

The R language for statistical computing was used for all data
analysis with extensive use of tidyverse packages for data
manipulation, mcr for deming regression, xts and tsa for time series
analysis (\hyperlink{citeproc_bib_item_6}{Team 2020}; \hyperlink{citeproc_bib_item_9}{Wickham et al. 2019}; \hyperlink{citeproc_bib_item_2}{Fabian Model <fabian.model@roche.com> 2014}; \hyperlink{citeproc_bib_item_4}{Ryan and Ulrich 2020}; \hyperlink{citeproc_bib_item_1}{Chan and Ripley 2020}). R scripts used for
data analysis are available here:
\url{https://github.com/hendersonmpa/biotinidase\_filter\_paper}

\section*{Results}
\label{sec:org3c7c2c2}
\subsection*{Time Series Analysis}
\label{sec:org51edebe}
Time series analysis was performed to examine trends in biotinidase
and GALT activity over the four year study period. As expected both
analytes demonstrate seasonal variation in measured activity, as a
result the screen positive rate for biotinidase deficiency and
galactosemia increased in the warm summer months (Figure \ref{fig:org3251666}
and \ref{fig:orgcf23ef4}). However, there was a period in early 2017 when the
screen positive rate for biotinidase deficiency increased despite
external temperatures below zero (Figure \ref{fig:org3251666} between the blue
lines). There is no corresponding change in the galactosemia screen
positive rate during this time period (Figure \ref{fig:org3251666} between the
blue lines).

% latex table generated in R 4.0.3 by xtable 1.8-4 package
% Tue Dec 14 20:31:37 2021
\begin{table}[ht]
\centering
\begin{tabular}{rrrrr}
  \hline
year & n & median & pos & rate \\ 
  \hline
2014 & 140620 & 119.87 &  78 & 0.00055 \\ 
  2015 & 140812 & 122.93 &  40 & 0.00028 \\ 
  2016 & 143361 & 120.25 &  38 & 0.00027 \\ 
  2017 & 144524 & 105.31 &  95 & 0.00066 \\ 
  2018 & 146365 & 111.90 &  88 & 0.0006 \\ 
  2019 & 83088 & 120.49 &  27 & 0.00032 \\ 
   \hline
\end{tabular}
\caption{Yearly Biotinidase Screen Positive Rate} 
\label{tab:biot_year}
\end{table}

% latex table generated in R 4.0.3 by xtable 1.8-4 package
% Tue Dec 14 20:31:38 2021
\begin{table}[ht]
\centering
\begin{tabular}{rrrrr}
  \hline
year & n & median & pos & rate \\ 
  \hline
2014 & 140678 & 8.37 &  20 & 0.00014 \\ 
  2015 & 140171 & 7.93 &  12 & 8.6e-05 \\ 
  2016 & 143352 & 8.13 &  21 & 0.00015 \\ 
  2017 & 143261 & 8.46 &  14 & 9.8e-05 \\ 
  2018 & 143592 & 8.22 &  13 & 9.1e-05 \\ 
  2019 & 82116 & 8.07 &   6 & 7.3e-05 \\ 
   \hline
\end{tabular}
\caption{Yearly Galactosemia Screen Positive Rate} 
\label{tab:galt_year}
\end{table}

\begin{figure}[htbp]
\centering
\includegraphics[width=\textwidth]{./figures/biotpts.pdf}
\caption{\label{fig:org3251666}Time series of weekly screen positive biotinidase deficiency referrals, median weekly biotinidase activity and mean weekly temperature (\degree{}C)}
\end{figure}

\begin{figure}[htbp]
\centering
\includegraphics[width=\textwidth]{./figures/galtpts.pdf}
\caption{\label{fig:orgcf23ef4}Time series of weekly screen positive galactosemia referrals, median weekly GALT activity and mean weekly temperature (\degree{}C)}
\end{figure}

\clearpage

\subsection*{Season and Trend Decomposition}
\label{sec:orgd2eac43}
Season and trend decomposition was used to identify trends in
biotinidase activity after adjustment for seasonal effects (Figure
\ref{fig:orgd5520fe}). Median weekly biotinidase activity revealed a marked
seasonal effect with higher activity in the winter (median + \(\le\) 17
MRU) and lower activity in the summer (median - \(\le\) 20) in addtion to
a non-linear negative trend during 2017 (Figure \ref{fig:orgd5520fe}). External
temperature was correlated with biotinidase activity (Pearson's r =
0.81) but not with the observed negative trend (Pearson's r = 0.004).

\begin{figure}[htbp]
\centering
\includegraphics[width=\textwidth]{./figures/biotdecomp.pdf}
\caption{\label{fig:orgd5520fe}Decomposition of the median weekly biotinidiase activity time series into seasonal, random and trend components.}
\end{figure}

\clearpage

\subsection*{Biotinidase Activity by Filter Paper Lot}
\label{sec:org46eeb1e}
Time series analysis of median weekly biotinidase results grouped by
filter paper lot revealed that filter paper cards exhibit a time
dependent inhibition of biotinidase activity, observed in seven filter
paper lots over four years (Figure \ref{fig:org42b1ffa}).

\begin{figure}[htbp]
\centering
\includegraphics[width=\textwidth]{./figures/biotform.pdf}
\caption{\label{fig:org42b1ffa}Median weekly biotinidase activity by filter paper collection card lot. Each lot of filter paper is plotted independently and indicated with a distinct colour.}
\end{figure}

\clearpage

\subsection*{Inhibition of Biotinidase Resolved After 2 Months}
\label{sec:org9344b79}
Direct comparison of measured biotinidase activity for whole blood
samples spotted onto a paper lot (w171) recently received from the
printers with material in circulation shows a notable negative
bias (Figure \ref{fig:orgee3cbff}). Inhibition of biotindase activity resolved after two months
(Figures \ref{fig:org42b1ffa} and \ref{fig:orgae22985}).

\begin{figure}[htbp]
\centering
\includegraphics[width=.9\linewidth]{./figures/spmat.pdf}
\caption{\label{fig:orgee3cbff}Comparison of biotinidase activity in samples collected simultaneously on filter paper lots w152, w161, w162, w171 at Time 0. linear regression- blue, line of identity- red}
\end{figure}

\begin{figure}[htbp]
\centering
\includegraphics[width=.9\linewidth]{./figures/demingw171.pdf}
\caption{\label{fig:org107ef6b}Biotinidase Activity on Filter Paper Lot W152 v W171 with Deming Regression at Time 0.}
\end{figure}
\clearpage

\begin{figure}[htbp]
\centering
\includegraphics[width=.9\linewidth]{./figures/spmat2.pdf}
\caption{\label{fig:orgae22985}Comparison of biotinidase activity in samples collected simultaneously on filter paper lots w152, w161, w162, w171 at Time 0 + 2 months. linear regression- blue, line of identity- red}
\end{figure}

\begin{figure}[htbp]
\centering
\includegraphics[width=.9\linewidth]{./figures/demingw171_2.pdf}
\caption{\label{fig:orgcd32bb7}Biotinidase Filter Paper W152 v W117 Lot Comparison at Time 0 + 2 months.}
\end{figure}

\clearpage

\section*{{\bfseries\sffamily TODO} Discussion}
\label{sec:org246a0b5}
This should explore the significance of the results of the work, not
repeat them. A combined Results and Discussion section is often
appropriate. Avoid extensive citations and discussion of published
literature.
\section*{Conclusions}
\label{sec:org85cd4ea}
In Ontario, new batches of filter paper collection cards are
generally printed and issued to birthing centres once a year in
early July. Introduction of three batches of filter paper cards in a
short period of time caused a pronounced negative trend in
biotinidase activity leading to an increased biotinidase deficiency
screen positive rate in 2017. Filter paper collection cards that
have been sequestered for two months prior to use no longer inhibit
biotinidase activity.

\section*{References}
\label{sec:orgd4f6b0e}
\begin{hangparas}{1.5em}{1}
\hypertarget{citeproc_bib_item_1}{Chan, Kung-Sik, and Brian Ripley. 2020. \textit{TSA: Time Series Analysis}. \href{https://CRAN.R-project.org/package=TSA}{https://CRAN.R-project.org/package=TSA}.}

\hypertarget{citeproc_bib_item_2}{Fabian Model <fabian.model@roche.com>, Ekaterina Manuilova Andre Schuetzenmeister <andre.schuetzenmeister@roche.com>. 2014. \textit{Mcr: Method Comparison Regression}. \href{https://CRAN.R-project.org/package=mcr}{https://CRAN.R-project.org/package=mcr}.}

\hypertarget{citeproc_bib_item_3}{Kwon, Charles, and Philip M. Farrell. 2000. “The magnitude and challenge of false-positive newborn screening test results.” \textit{Archives of Pediatrics and Adolescent Medicine} 154 (7):714–18. \href{https://doi.org/10.1001/archpedi.154.7.714}{https://doi.org/10.1001/archpedi.154.7.714}.}

\hypertarget{citeproc_bib_item_4}{Ryan, Jeffrey A., and Joshua M. Ulrich. 2020. \textit{Xts: eXtensible Time Series}. \href{https://CRAN.R-project.org/package=xts}{https://CRAN.R-project.org/package=xts}.}

\hypertarget{citeproc_bib_item_5}{Strovel, Erin T., Tina M. Cowan, Anna I. Scott, and Barry Wolf. 2017. “Laboratory diagnosis of biotinidase deficiency, 2017 update: A technical standard and guideline of the American College of Medical Genetics and Genomics.” \textit{Genetics in Medicine} 19 (10). Nature Publishing Group. \href{https://doi.org/10.1038/gim.2017.84}{https://doi.org/10.1038/gim.2017.84}.}

\hypertarget{citeproc_bib_item_6}{Team, R Core. 2020. \textit{R: A Language and Environment for Statistical Computing}. Vienna, Austria: R Foundation for Statistical Computing. \href{https://www.R-project.org/}{https://www.R-project.org/}.}

\hypertarget{citeproc_bib_item_7}{Thibodeau, Deborah L., Wanda Andrews, Joanne Meyer, Paige Mitchell, and Barry Wolf. 1993. “Comparison of the effects of season and prematurity on the enzymatic newborn screening tests for galactosemia and biotinidase deficiency.” \textit{Screening} 2 (1):19–27. \href{https://doi.org/10.1016/0925-6164(93)90014-A}{https://doi.org/10.1016/0925-6164(93)90014-A}.}

\hypertarget{citeproc_bib_item_8}{Watson, Michael S., Marie Y. Mann, Michele A. Lloyd-Puryear, Piero Rinaldo, R. Rodney Howell, and American College of Medical Genetics Newborn Screening Expert Group. 2006. “Newborn Screening: Toward a Uniform Screening Panel and System—Executive Summary.” \textit{Pediatrics} 117 (Supplement\_3):S296–307. \href{https://doi.org/10.1542/peds.2005-2633I}{https://doi.org/10.1542/peds.2005-2633I}.}

\hypertarget{citeproc_bib_item_9}{Wickham, Hadley, Mara Averick, Jennifer Bryan, Winston Chang, Lucy D’Agostino McGowan, Romain François, Garrett Grolemund, et al. 2019. “Welcome to the tidyverse.” \textit{Journal of Open Source Software} 4 (43):1686. \href{https://doi.org/10.21105/joss.01686}{https://doi.org/10.21105/joss.01686}.}

\hypertarget{citeproc_bib_item_10}{Wolf, Barry. 2015. “The Story of Biotinidase Deficiency and Its Introduction into Newborn Screening: The Role of Serendipity.” \textit{International Journal of Neonatal Screening} 1 (1):3–12. \href{https://doi.org/10.3390/ijns1010003}{https://doi.org/10.3390/ijns1010003}.}
\end{hangparas}
\end{document}