% Created 2021-12-30 Thu 14:32
% Intended LaTeX compiler: pdflatex
\documentclass[review]{elsarticle}
\usepackage[utf8]{inputenc}
\usepackage[T1]{fontenc}
\usepackage{graphicx}
\usepackage{longtable}
\usepackage{wrapfig}
\usepackage{rotating}
\usepackage[normalem]{ulem}
\usepackage{amsmath}
\usepackage{amssymb}
\usepackage{capt-of}
\usepackage{hyperref}
\usepackage{lineno}
\linenumbers
\usepackage{setspace}
\onehalfspacing
\authblk
\usepackage{pdfpages}
\usepackage{textpos}
\usepackage[final]{draftwatermark}
\usepackage{gensymb}
\usepackage{amsmath}
\usepackage{chemfig}
\setchemfig{atom style={scale=0.45}}
\usepackage[]{mhchem}
\date{}
\title{}
\hypersetup{
 pdfauthor={Matthew Henderson},
 pdftitle={},
 pdfkeywords={},
 pdfsubject={},
 pdfcreator={Emacs 27.2 (Org mode 9.5)}, 
 pdflang={English}}
\usepackage[notquote]{hanging}
\begin{document}\makeatletter
\newcommand{\citeprocitem}[2]{\hyper@linkstart{cite}{citeproc_bib_item_#1}#2\hyper@linkend}
\makeatother



\begin{frontmatter}
\title{Seasonal Temperature and Filter Paper Collection Cards Affect Biotinidase Activity}
\author[NSO, UoO]{Matthew P.A. Henderson\corref{cor1}}
\ead{mhenderson@cheo.on.ca}
\author[NSO]{Nathan McIntosh}
\author[NSO]{Amy Chambers}
\author[NSO]{Emily Desormeaux}
\author[NSO]{Michael Kowalski}
\author[NSO]{Jennifer Milburn}
\author[NSO, UO]{Pranesh Chakraborty}
\address[NSO]{Newborn Screening Ontario, Children's Hospital of Eastern Ontario}
\address[UoO]{Department of Medicine, University of Ottawa} 
\cortext[cor1]{Corresponding author}
\end{frontmatter}

\section*{Abstract}
\label{sec:org10fc8ab}
\begin{description}
\item[{Objective}] This study set out to examine pre-analytical factors
affecting the frequency of positive results in newborn screening for
biotinidase deficiency. This investigation was prompted by an
increase in annual screen positive rate for biotinidase deficiency
in Ontario from 2.65x10\textsuperscript{-4} in 2016 to 6.57x10\textsuperscript{-4} in 2017.

\item[{Results}] Season and trend decomposition was used to separate
seasonality from an underlying trend in the time series of
biotindase activity measurements for the period 2014-01-12 to
2019-07-27 (n = 798,770). This analysis revealed a marked seasonal
effect (winter = median + \(\le\) 17 MRU, summer = mean - \(\le\) 20 MRU)
and a non-linear negative trend. Seasonal temperature was correlated
with biotinidase results (Pearson's r = 0.79) but not with the
observed negative trend (Pearson's r = 0.0025). Time series analysis
of biotinidase results grouped by print lot of filter paper revealed
that recently printed filter paper cards exhibit a time dependent
inhibition of biotinidase activity. This inhibition resolved over
time.

\item[{Conclusions}] Increased seasonal temperatures and collection on
newly printed filter cards inhibits biotindase activity.
\end{description}

\section*{Keywords}
\label{sec:org57a1784}
newborn screening, biotinidase deficiency, dried blood spots, filter
paper, seasonal variation, enzyme activity
\section*{Introduction}
\label{sec:org346d980}
This study set out to examine pre-analytical factors affecting the
frequency of positive results in newborn screening for biotinidase
deficiency. This investigation was prompted by a marked increase in
annual screen positive rate for biotinidase deficiency in Ontario from
2.65x10\textsuperscript{-4} in 2016 to 6.57x10\textsuperscript{-4} in 2017.

Biotinidase deficiency is an autosomal recessive disorder caused by
absent or markedly deficient activity of biotinidase.  Biotinidase is
a cytosolic enzyme that liberates biotin from biocytin during the
normal proteolytic turnover of holocarboxylases and other biotinylated
proteins. The absence of appropriate biotin recycling is associated
with secondary alterations in amino acid, carbohydrate, and fatty acid
metabolism (Figure \ref{fig:org3908b43}). Neurological symptoms such as lethargy,
muscular hypotonia, grand mal and myoclonic seizures, and ataxia may
occur in the neonatal period. Many children also have developmental
delay, hearing loss, conjunctivitis and visual problems including
optic atrophy. Metabolic acidosis and the characteristic organic
aciduria of multiple carboxylase deficiency is often absent in the
early stages of the disease.

The state of Virginia was the first jurisdiction to perform universal
newborn screening for biotinidase deficiency, beginning in 1984 (\hyperlink{citeproc_bib_item_12}{Wolf 2015}). Biotinidase deficiency was added to the recommended
universal screening panel in the United States in 2006 and has been a
target disorder for newborn screening in Ontario since 2007 (\hyperlink{citeproc_bib_item_10}{Watson et al. 2006}).  Based on newborn screening outcome data from
2006, the incidence of profound biotinidase deficiency in the United
States is estimated at 1/80,000 births, and partial biotinidase
deficiency between 1/31,000 and 1/40,000 (\hyperlink{citeproc_bib_item_7}{Strovel et al. 2017}). Diagnosis of biotinidase deficiency is based on
demonstrating deficient enzyme activity in serum or plasma. Patients
with profound biotinidase deficiency have less than 10\% of mean normal
serum activity (\hyperlink{citeproc_bib_item_7}{Strovel et al. 2017}). Patients with partial biotinidase
deficiency have 10-30\% of mean normal serum activity (\hyperlink{citeproc_bib_item_7}{Strovel et al. 2017}).

Avoiding false negative newborn screening results is of paramount
importance in newborn screening. There is recognition that false
positive newborn screening results are also an adverse outcome of
newborn screening and need to be reduced for the benefit of families
and the healthcare system (\hyperlink{citeproc_bib_item_4}{Kwon and Farrell 2000}). A previous observational
study examining the affect of season on the screen positive rate for
biotinidase deficiency found no significant seasonal effect, this may
be due to differences in climate and screening thresholds (\hyperlink{citeproc_bib_item_9}{Thibodeau et al. 1993}). Controlled experiments demonstrate that
exposure to high temperature and humidity decreases biotinidase
activity (\hyperlink{citeproc_bib_item_1}{Adam et al. 2011}). Evidence presented here demonstrates that
increased seasonal temperatures and collection on newly printed filter
cards inhibits biotindase activity results in an increased screen
positive rate.



\begin{figure}[htbp]
\centering
\includegraphics[width=.8\textwidth]{./figures/cycle.pdf}
\caption{\label{fig:org3908b43}The Biotin Cycle adapted from (\hyperlink{citeproc_bib_item_7}{Strovel et al. 2017}). Grey ovals contain substrates and products. Labelled black points represent enzymes. Metabolic functions of holocarboxylases are listed in the octagon.}
\end{figure}

\section*{Material and Methods}
\label{sec:org92bcc06}
A data mining approach was used to investigate the affect of
temperature and filter paper lot on the screen positive rate for
biotinidase deficiency. The laboratory information system was queried
for all newborn screening biotinidase activity results from the first
phase of screening for the period 2014-01-12 to 2019-07-27 (n =
798,770). During the study period biotinidase activity was measured on
the SpotCheck Pro platform (Astoria Pacific, Oregon USA.). Biotinidase
activity was reported in microplate response units (MRU).

Daily mean temperature data for weather station 42183 in Barrie
Ontario was downloaded from Environment Canada. Season and trend
decomposition was used to separate seasonality from an underlying
trend in the 4 year time series of biotindase activity results.

Whole blood samples were heat treated to reduce biotinidase
activity. The heat treated whole blood was then mixed with untreated
whole blood in a serial dilution to create a set of samples with a
range of biotinidase activity. Biotinidase acitivty in these samples
was measured after spotting onto filter paper lots W152, W161, W162
and W171 and allowing to dry at room temperature overnight.

This manuscript was prepared using the Org-mode environment for
literate programming and reproducible research (\hyperlink{citeproc_bib_item_6}{Schulte et al. 2012}). The R language for statistical computing was used
for all data analysis with tidyverse packages for
data manipulation, mcr for deming regression, xts and tsa for time
series analysis (\hyperlink{citeproc_bib_item_8}{Team 2020}; \hyperlink{citeproc_bib_item_11}{Wickham et al. 2019}; \hyperlink{citeproc_bib_item_3}{Fabian Model <fabian.model@roche.com> 2014}; \hyperlink{citeproc_bib_item_5}{Ryan and Ulrich 2020}; \hyperlink{citeproc_bib_item_2}{Chan and Ripley 2020}). R scripts
used for data analysis are available here:
\url{https://github.com/hendersonmpa/biotinidase\_filter\_paper}

\section*{Results}
\label{sec:org8f773d1}
\subsection*{Time Series Analysis}
\label{sec:org6327398}
Time series analysis was performed to examine trends in biotinidase
and GALT activity over the four year study period. As expected both
analytes demonstrate seasonal variation in measured activity, as a
result the screen positive rate for biotinidase deficiency and
galactosemia increased in the warm summer months (Figure \ref{fig:org00af45b}
and \ref{fig:orgd80f6e1}). However, there was a period in early 2017 when the
screen positive rate for biotinidase deficiency increased despite
external temperatures below zero (Figure \ref{fig:org00af45b} between the blue
lines). There is no corresponding change in the galactosemia screen
positive rate during this time period (Figure \ref{fig:org00af45b} between the
blue lines).

% latex table generated in R 4.0.3 by xtable 1.8-4 package
% Thu Dec 30 14:32:39 2021
\begin{table}[ht]
\centering
\begin{tabular}{rrrrr}
  \hline
year & n & median & pos & rate \\ 
  \hline
2014 & 140620 & 119.87 &  78 & 5.55e-04 \\ 
  2015 & 140812 & 122.93 &  40 & 2.84e-04 \\ 
  2016 & 143361 & 120.25 &  38 & 2.65e-04 \\ 
  2017 & 144524 & 105.31 &  95 & 6.57e-04 \\ 
  2018 & 146365 & 111.90 &  88 & 6.01e-04 \\ 
   \hline
\end{tabular}
\caption{Yearly Biotinidase Screen Positive Rate} 
\label{tab:biot_year}
\end{table}

% latex table generated in R 4.0.3 by xtable 1.8-4 package
% Thu Dec 30 14:32:39 2021
\begin{table}[ht]
\centering
\begin{tabular}{rrrrr}
  \hline
year & n & median & pos & rate \\ 
  \hline
2014 & 140678 & 8.37 &  20 & 1.42e-04 \\ 
  2015 & 140171 & 7.93 &  12 & 8.56e-05 \\ 
  2016 & 143352 & 8.13 &  21 & 1.46e-04 \\ 
  2017 & 143261 & 8.46 &  14 & 9.77e-05 \\ 
  2018 & 143592 & 8.22 &  13 & 9.05e-05 \\ 
   \hline
\end{tabular}
\caption{Yearly Galactosemia Screen Positive Rate} 
\label{tab:galt_year}
\end{table}

\begin{figure}[htbp]
\centering
\includegraphics[width=\textwidth]{./figures/biotpts.pdf}
\caption{\label{fig:org00af45b}Time series of weekly screen positive biotinidase deficiency referrals, median weekly biotinidase activity and mean weekly temperature (\degree{}C)}
\end{figure}

\begin{figure}[htbp]
\centering
\includegraphics[width=\textwidth]{./figures/galtpts.pdf}
\caption{\label{fig:orgd80f6e1}Time series of weekly screen positive galactosemia referrals, median weekly GALT activity and mean weekly temperature (\degree{}C)}
\end{figure}

\clearpage

\subsection*{Season and Trend Decomposition}
\label{sec:orga2ab1b8}
Season and trend decomposition was used to identify trends in
biotinidase activity after adjustment for seasonal effects (Figure
\ref{fig:orgc23c2c6}). This analysis revealed a marked seasonal effect with
higher activity in the winter (median + \(\le\) 17 MRU) and lower activity
in the summer (median - \(\le\) 20 MRU) and a non-linear negative trend
during 2017 and 2018 (Figure \ref{fig:orgc23c2c6}). External temperature was
correlated with biotinidase activity (Pearson's r = 0.79) but not with
the observed negative trend (Pearson's r = 0.025).

\begin{figure}[htbp]
\centering
\includegraphics[width=\textwidth]{./figures/biotdecomp.pdf}
\caption{\label{fig:orgc23c2c6}Decomposition of the median weekly biotinidiase activity time series into seasonal, random and trend components.}
\end{figure}

\clearpage

\subsection*{Biotinidase Activity by Filter Paper Lot}
\label{sec:org0c05a8a}
Time series analysis of median weekly biotinidase results grouped by
filter paper lot revealed that filter paper cards exhibited a time
dependent inhibition of biotinidase activity, observed in seven filter
paper lots over four years (Figure \ref{fig:org321fef1}). Due to a time sensitive
change required in the filter paper collection card, lot W161 was put into
circulation soon after printing. Inhibition of biotinidase activity in
this card lot was most pronounced and took over 5 months to resolve
(Figure \ref{fig:org321fef1}, yellow line and Table \ref{tab:w161_months}).

\begin{figure}[htbp]
\centering
\includegraphics[width=\textwidth]{./figures/biotform.pdf}
\caption{\label{fig:org321fef1}Median weekly biotinidase activity by filter paper collection card lot. Each lot of filter paper is plotted independently and indicated with a distinct colour.}
\end{figure}

\clearpage

% latex table generated in R 4.0.3 by xtable 1.8-4 package
% Thu Dec 30 14:32:43 2021
\begin{table}[ht]
\centering
\begin{tabular}{lrrrr}
  \hline
Month & Median & Median W161 & Difference & \% Difference \\ 
  \hline
February & 137.3 & 57.6 & -79.8 & -58.1 \\ 
  March & 145.2 & 63.8 & -81.4 & -56.1 \\ 
  April & 135.4 & 69.7 & -65.7 & -48.5 \\ 
  May & 128.2 & 74.8 & -53.3 & -41.6 \\ 
  June & 120.2 & 79.4 & -40.8 & -33.9 \\ 
  July & 117.7 & 92.2 & -25.5 & -21.7 \\ 
  August & 111.3 & 102.1 & -9.3 & -8.3 \\ 
  September & 119.0 & 107.5 & -11.5 & -9.7 \\ 
  October & 124.0 & 122.6 & -1.4 & -1.2 \\ 
  November & 135.4 & 137.8 & 2.4 & 1.8 \\ 
  December & 132.3 & 135.7 & 3.3 & 2.5 \\ 
   \hline
\end{tabular}
\caption{Median Activity for Filter Card Lot W161} 
\label{tab:w161_months}
\end{table}

\subsection*{Inhibition of Biotinidase Resolved After 2 Months}
\label{sec:orgd2b3160}
Based on the observed inhibition of biotinidase activity with filter
paper lot W161 (Figure \ref{fig:org321fef1}, yellow line and Table
\ref{tab:w161_months}) a controlled experiment was conducted. Whole
blood samples were spotted onto a new filter lot (W171) and filter
paper lots in circulation (W152, W161, W162). Lot 171 showed a notable
negative bias (Figure \ref{fig:orga9eed72}). Inhibition of biotindase activity by
filter paper lot W171 resolved after two months of storage in the
laboratory (Figure \ref{fig:org13533e3}).

\begin{figure}[htbp]
\centering
\includegraphics[width=.9\linewidth]{./figures/spmat.pdf}
\caption{\label{fig:orga9eed72}Comparison of biotinidase activity (MRU) in samples collected simultaneously on filter paper lots w152, w161, w162, w171 at Time 0. linear regression- blue, line of identity- red}
\end{figure}

\begin{figure}[htbp]
\centering
\includegraphics[width=.9\linewidth]{./figures/demingw171.pdf}
\caption{\label{fig:orgf5bbff8}Biotinidase Activity (MRU) on Filter Paper Lot W152 v W171 with Deming Regression at Time 0.}
\end{figure}
\clearpage

\begin{figure}[htbp]
\centering
\includegraphics[width=.9\linewidth]{./figures/spmat2.pdf}
\caption{\label{fig:org13533e3}Comparison of biotinidase activity (MRU) in samples collected simultaneously on filter paper lots w152, w161, w162, w171 at Time 0 + 2 months. linear regression- blue, line of identity- red}
\end{figure}

\begin{figure}[htbp]
\centering
\includegraphics[width=.9\linewidth]{./figures/demingw171_2.pdf}
\caption{\label{fig:org141fd8c}Biotinidase Activity (MRU) on Filter Paper Lot W152 v W171 with Deming Regression at Time 0 + 2 months.}
\end{figure}

\clearpage


\section*{Discussion}
\label{sec:org9058b39}
This study demonstrates that seasonal temperature and filter paper
collection cards affect biotinidase activity. A previous study has
used controlled experiments to showed that storage at high temperature
and humidity decrease biotinidase activity (\hyperlink{citeproc_bib_item_1}{Adam et al. 2011}). However a
previous observational study examining the affect of season on the
screen positive rate for biotinidase deficiency found no significant
seasonal effect, this may be due to differences in climate and
screening thresholds (\hyperlink{citeproc_bib_item_9}{Thibodeau et al. 1993}).

Our program has been aware of seasonal variation in biotinidase
activity for some time. It was thought that elevated temperature and
humidity was the cause of the observed annual increase in screen
positive rate for biotinidase deficiency during the summer. A notable
increase in screen positive rate coinciding with the rapid
distribution of new filter paper cards in the winter of 2017 prompted
us to examine the effect of filter paper collection cards on
biotinidase activity. Once we examined the relationship between filter
paper lot and biotinidase activity a clear time dependent relationship
was evident. Unfortunately, because we distribute new lots of filter
paper cards to birthing centres in July the effect of the filter paper
cards on biotinidase activity was attributed to seasonal variation.

We have shown that the effect of filter paper lot on biotindase
activity decreases over time. Future controlled studies should examine
the role water content in the filter paper and chemicals used in the
printing process have on biotinidase activity. In the meantime our
program is sequestering filter paper collection cards for two months
prior to release to birthing centres.

\section*{Conclusions}
\label{sec:orgcb4bae7}
In Ontario, new batches of filter paper collection cards are generally
printed and issued to birthing centres once a year in early
July. Introduction of three batches of filter paper cards in a short
period of time caused a pronounced negative trend in biotinidase
activity leading to an increased biotinidase deficiency screen
positive rate in 2017 and 2018. Filter paper collection cards that
have been stored for two months prior to use no longer inhibit
biotinidase activity.

\section*{References}
\label{sec:org73b26e5}
\begin{hangparas}{1.5em}{1}
\hypertarget{citeproc_bib_item_1}{Adam, B. W., E. M. Hall, M. Sternberg, T. H. Lim, S. R. Flores, S. O’Brien, D. Simms, L. X. Li, V. R. De Jesus, and W. H. Hannon. 2011. “The stability of markers in dried-blood spots for recommended newborn screening disorders in the United States.” \textit{Clinical Biochemistry} 44 (17-18):1445–50. \href{https://doi.org/10.1016/j.clinbiochem.2011.09.010}{https://doi.org/10.1016/j.clinbiochem.2011.09.010}.}

\hypertarget{citeproc_bib_item_2}{Chan, Kung-Sik, and Brian Ripley. 2020. \textit{TSA: Time Series Analysis}. \href{https://CRAN.R-project.org/package=TSA}{https://CRAN.R-project.org/package=TSA}.}

\hypertarget{citeproc_bib_item_3}{Fabian Model <fabian.model@roche.com>, Ekaterina Manuilova Andre Schuetzenmeister <andre.schuetzenmeister@roche.com>. 2014. \textit{Mcr: Method Comparison Regression}. \href{https://CRAN.R-project.org/package=mcr}{https://CRAN.R-project.org/package=mcr}.}

\hypertarget{citeproc_bib_item_4}{Kwon, Charles, and Philip M. Farrell. 2000. “The magnitude and challenge of false-positive newborn screening test results.” \textit{Archives of Pediatrics and Adolescent Medicine} 154 (7):714–18. \href{https://doi.org/10.1001/archpedi.154.7.714}{https://doi.org/10.1001/archpedi.154.7.714}.}

\hypertarget{citeproc_bib_item_5}{Ryan, Jeffrey A., and Joshua M. Ulrich. 2020. \textit{Xts: eXtensible Time Series}. \href{https://CRAN.R-project.org/package=xts}{https://CRAN.R-project.org/package=xts}.}

\hypertarget{citeproc_bib_item_6}{Schulte, Eric, Dan Davison, Thomas Dye, and Carsten Dominik. 2012. “A Multi-Language Computing Environment for Literate Programming and Reproducible Research.” \textit{Journal of Statistical Software} 46 (3):1–11. \href{https://doi.org/10.18637/jss.v046.i03}{https://doi.org/10.18637/jss.v046.i03}.}

\hypertarget{citeproc_bib_item_7}{Strovel, Erin T., Tina M. Cowan, Anna I. Scott, and Barry Wolf. 2017. “Laboratory diagnosis of biotinidase deficiency, 2017 update: A technical standard and guideline of the American College of Medical Genetics and Genomics.” \textit{Genetics in Medicine} 19 (10). Nature Publishing Group. \href{https://doi.org/10.1038/gim.2017.84}{https://doi.org/10.1038/gim.2017.84}.}

\hypertarget{citeproc_bib_item_8}{Team, R Core. 2020. \textit{R: A Language and Environment for Statistical Computing}. Vienna, Austria: R Foundation for Statistical Computing. \href{https://www.R-project.org/}{https://www.R-project.org/}.}

\hypertarget{citeproc_bib_item_9}{Thibodeau, Deborah L., Wanda Andrews, Joanne Meyer, Paige Mitchell, and Barry Wolf. 1993. “Comparison of the effects of season and prematurity on the enzymatic newborn screening tests for galactosemia and biotinidase deficiency.” \textit{Screening} 2 (1):19–27. \href{https://doi.org/10.1016/0925-6164(93)90014-A}{https://doi.org/10.1016/0925-6164(93)90014-A}.}

\hypertarget{citeproc_bib_item_10}{Watson, Michael S., Marie Y. Mann, Michele A. Lloyd-Puryear, Piero Rinaldo, R. Rodney Howell, and American College of Medical Genetics Newborn Screening Expert Group. 2006. “Newborn Screening: Toward a Uniform Screening Panel and System-Executive Summary.” \textit{Pediatrics} 117 (Supplement\_3):S296–307. \href{https://doi.org/10.1542/peds.2005-2633I}{https://doi.org/10.1542/peds.2005-2633I}.}

\hypertarget{citeproc_bib_item_11}{Wickham, Hadley, Mara Averick, Jennifer Bryan, Winston Chang, Lucy D’Agostino McGowan, Romain François, Garrett Grolemund, et al. 2019. “Welcome to the tidyverse.” \textit{Journal of Open Source Software} 4 (43):1686. \href{https://doi.org/10.21105/joss.01686}{https://doi.org/10.21105/joss.01686}.}

\hypertarget{citeproc_bib_item_12}{Wolf, Barry. 2015. “The Story of Biotinidase Deficiency and Its Introduction into Newborn Screening: The Role of Serendipity.” \textit{International Journal of Neonatal Screening} 1 (1):3–12. \href{https://doi.org/10.3390/ijns1010003}{https://doi.org/10.3390/ijns1010003}.}
\end{hangparas}
\end{document}